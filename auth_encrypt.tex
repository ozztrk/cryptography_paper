\section{Authenticated Encryption with Associated Data}
% Questions:
% what is S_r and S_c?
% What is a Nonce And Tag?
After many years of relying on traditional methods for encryption and authentication through 'generic composition,' new constructions have emerged over the past two decades. They achieved both privacy and authenticity simultaneously and often more efficiently than solutions using generic composition.
\newline
In many environments we do not only have to encrypt and authenticate the message or payload, but also wish to include auxiliary data (like the header of a network packet) which should be authenticated, but left unencrypted. \cite[Chapter 1]{Black2005}
% Umschreiben
The reason being that routers must be able to read the headers of packets in order to know how to properly route them. This need spurred some designers of AE schemes to allow “associated data” to be included as input to their schemes. Such schemes have been termed AEAD schemes (Authenticated Encryption with Associated Data) which rely on symmetric key cryptography.\footnote[1]{\textbf{Symmetric key cryptography}, also called secret or single-key cryptography, uses the same key to both encrypt and decrypt information and is used primarily to ensure data confidentiality. In cases where the information is encrypted and decoded by the same person, there is no need to share the secret key. However, when these operations involve different people or equipment, it is necessary that the secret key be previously combined through a secure communication channel. Examples of cryptographic methods that use a symmetric key are: DES, AES, Blowfish, RC4, 3DES, and IDEA. \cite{Alencar2022Cryptography}}
A notion which was first formalized by Rogaway \cite{10.1145/586110.586125}.

\subsection{Ascon AEAD Scheme}
In this section, we dive into the Ascon-128 scheme, which, along with Ascon-128a, has been designated as the “primary choice” for lightweight authenticated encryption in the final portfolio of the CAESAR competition. These schemes are notable for their robust security, with no known vulnerabilities. The most effective attacks target only the versions with reduced rounds—specifically, the initialization reduced to 7 out of 12 rounds—and these are still far from posing a practical threat. Notably, benchmarks have demonstrated that Ascon excels in efficiency, particularly for short messages, confirming its status as a state-of-the-art lightweight encryption method. \cite[Chapter 1]{Ascon-v1.2}


\subsection{Authenticated Encryption $E_{k,r,a,b}$}
The encryption procedure $E_{k,r,a,b}$ takes the following inputs:
\begin{itemize}
    \item Secret key $K$ with $k$ bits.
    % find Citation! and make footnote!
    \item Nonce (public message number) $N$ with 128 bits. \footnote[2]{\textbf{A nonce} is a random or non-repeating value that is included in data
    exchanged by a protocol, usually for the purpose of guaranteeing
    liveness and thus detecting and protecting against replay attacks.\cite{rfc4949}} % paraphrase
    \item Associated data $A$ of arbitrary length. (f.e. the header of a networkpacket)
    \item Plaintext $P$ of arbitrary length. (the payload)
\end{itemize}
It produces an output consisting of the authenticated ciphertext $C$ of exactly the same length as the plaintext $P$ plus an authentication tag $T$ of size 128 bits, which authenticates both the associated data and the encrypted message:

\[
E_{k,r,a,b}(K,N,A,P) = (C,T)
\]
Here, $A$ is not encrypted but authenticated. 
The recommended parameterset for Ascon-128 is $E_{126,64,16,6}$ and for Ascon-128a $b$ is 8. \cite[Chapter 2.2]{Ascon-v1.2} \newline
Ascon is natively defined on 64-bit words using only the bitwise
Boolean functions and, xor, not, and rot (bitwise rotation). This significantly
reduces the effort of implementing the algorithm on new target platforms. \cite[Chapter 4.1]{Ascon-v1.2} % paraphrase this part too

\subsection{Decryption and Verification $D_{k,r,a,b}$}
The decryption and verification procedure $D_{k,r,a,b}$ takes the following inputs:
\begin{itemize}
    \item Key $K$.
    \item Nonce $N$.
    \item Associated data $A$.
    \item Ciphertext $C$.
    \item Tag $T$.
\end{itemize}
It outputs either the plaintext $P$ if the verification of the tag is correct, or an error $\bot$ if the verification of the tag fails:
\[
D_{k,r,a,b}(K,N,A,C,T) \in \{P, \bot\}
\]
\cite{Ascon-v1.2}

\subsection{Algorithm} % TODO: Find duplex mode citation
Ascons approach for authenticated encrytion is based on duplex modes like MonkeyDuplex \footnote[3]{Find References for Duplex Mode and stuff}, but with stronger keyed initialization and keyed finalization function. The encryption and decryption operations are depicted in Figure 2. \cite{Ascon-v1.2}
\begin{figure}[H]
    \centering
    \includegraphics[width=1\textwidth]{figures/aead-algorithm.png}
    \caption{Ascon's mode of operation \cite{Ascon-v1.2}}
    \label{fig:aead-algorithm}
\end{figure}
\subsubsection{Initialization}
In contrary to other sponge-based authenticated encryption schemes, Ascon uses stronger keyed initialization and keyed finalization phases. This makes sure that if an attacker manages to recover the internal state during data processing (f.e. with side-channel attacks), the attacker is not directly able to recover the secret key.\cite[Chapter 5.1]{Ascon-v1.2}
"The 320-bit initial state of Ascon is formed by [concatenating] the secret key K of k bits and nonce N of 128 bits, as well as an IV specifying the algorithm (including the key size k, the rate r, the initialization and finalization round number a, and the intermediate round number b, each written as an 8-bit integer)", followed by padding of zeros for the required length. % maybe paraphrase
$$S \leftarrow k || r || a || b || 0^{160-k} || K || N $$
In the Initialization $a$ rounds of the permutation $p$ is applied to the initial state. After that, the resulting state is XORed with a modified version of the secret key $K$ which is prepended with a series of $320-k$ zeros to pad it to the size of 320 bits. \cite{Ascon-v1.2}
$$S \leftarrow p^a(S) \oplus (0^{320-k} || K)$$
This step is needed to... 
\subsubsection{Processing Associated Data}
If there is associated data, it will first be preprocessed for absorbtion into blocks of $r$ bits which will be appended by a single $1$ and the smallest number of zeros to obtain a multiple of $r$ and then split again in $s$ blocks of $r$ bits. If there is no associated data, no padding is apllied and $s=0$.
The associated data $A$ is then absorbed into the state.
$$ S \leftarrow p^b((S_r \oplus A_i) || S_c),\ \ 1 \leq i \leq s$$
After all associated data is absorbed, a domain separation constant is added to differentiate the processing of associated data from the encryption of the payload. 
\cite[Chapter 2.4.2]{Ascon-v1.2} \newline
%paraphrase
The addition of $0^{319} \Vert 1$ after the last processed associated data block and the first plaintext block acts as a domain separation to prevent attacks that change the role of plaintext and associated data blocks. If no associated data and only an incomplete plaintext block is present, the two initialization and finalization calls to pa are sufficient and no additional intermediate round transformation pb is needed. To prevent that key additions between the two applications of pa cancel each other out in this case, they are added to different parts of the capacity part Sc of the state. \cite[Chapter 5.1.1]{Ascon-v1.2}
$$S \leftarrow S \oplus (0^{319} || 1)$$

\subsubsection{Processing Plaintext/Ciphertext}
As for the payload it is also processed into blocks the same way. 
$$P_1, \dots , P_t \leftarrow \ r-bit blocks of P || 1 || 0^{r-1-(|P| mod\ r)}$$
For encryption each padded plaintext block $P_i$ is XORed to the first $r$ bits $S_r$ of the state $S$ and for each block except the last one is it is transformed by the permutation $p^b$% why???
$$C_i \leftarrow S_r \oplus P_i$$
$$
s \leftarrow \begin{cases} 
    p^b(C_i \parallel S_c) & \text{if } 1 \leq i < t \\
    C_i \parallel S_c & \text{if } 1 \leq i = t
    \end{cases}
$$
The last remaining block $C_t$ the padding is removed. This ensures that the decrypted plaintext matches the original plaintext exactly, without any additional padding bytes affecting the integrity of the data.
$$\tilde{C}_t \leftarrow \lfloor C_t \rfloor_{|P| \text{ mod } r}$$
\newline
To decrypt the ciphertext back the its original, each ciphertextblock except the last one is being XORed with the first $r$ bits $S_r$ of the internal state, which then has to be replaced by $C_i$ and then transformed by the $b$-round permutation $p^b$:
$$ P_i\leftarrow S_r \oplus C_i $$
$$ S \leftarrow (S_r \oplus (\tilde P_t || 1 || 0^{r-1-l})) || S_c $$
For the remaining truncated ciphertextblock $\tilde C_t\ with\ 0 \leq l \le r$ bits, the procedure differs:
$$\tilde{P}_t \leftarrow \lfloor S_r \rfloor_\ell \oplus \tilde{C}_t$$
$$S \leftarrow (S_r \oplus (\tilde P_t || 1 || 0^{r-1-l})) || S_c$$
\cite{Ascon-v1.2}
\subsubsection{Finalization}
The Finalization step, is genereting the tag which is important for \dots
The secret Key $K$ is first XORed to the internal state which then is transformed by the permutation $p^a$ using $a$. The tag $T$ consists of the last 128 bits of the state XORed with the last 128 bits of the key $K$: %why last 128 bits isnt the key exactly that long?
$$S \leftarrow p^a(S \oplus (0^r || K || 0^{c-k}))$$
$$ T \leftarrow \lceil S  \rceil^{128} \oplus \lceil K \rceil^{128}$$
The encryption procedure is returning the tag $T$ together with the ciphertext $C_1||...||\tilde C_t$ and the decription procedure is just returning the plaintext (payload) if the calculated tag value matches the received tag value $T$. \cite{Ascon-v1.2}

\subsection{Principles Behind Ascon's Cryptographic Techniques}
\dots
