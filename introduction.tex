\section{Introduction}
The rise of the Internet of Things (IoT), embedded systems, and other constrained environments has necessitated the development of lightweight cryptographic solutions. These solutions must balance the demands for security, efficiency, and minimal resource consumption. Traditional cryptographic algorithms are often too resource-intensive for these applications, making lightweight cryptography an essential area of research.

This paper explores the key challenges and approaches in lightweight cryptography, focusing on three main areas: authenticated encryption, hashing, and permutation algorithms. Each section provides an in-depth analysis of current techniques and their applicability to resource-constrained environments. 

In the realm of authenticated encryption, we examine modern schemes that ensure both data confidentiality and integrity with minimal overhead. Our discussion includes an evaluation of the Ascon AEAD scheme, recognized for its efficiency and security in constrained environments. For hashing, we delve into lightweight hash functions designed to provide robust security while maintaining low computational costs. Lastly, we explore permutation-based cryptographic primitives, which offer an alternative approach to constructing secure and efficient cryptographic algorithms.

Through this comprehensive review, we aim to highlight the advancements and ongoing research in lightweight cryptography, providing a valuable resource for both researchers and practitioners working to secure the next generation of connected devices.
