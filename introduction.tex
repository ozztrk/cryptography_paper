\section{Introduction}
Lightweight cryptography is a subfield of cryptographic research focusing on securing constrained environments, such as embedded systems, RFID tags, and IoT devices. These environments demand cryptographic algorithms that are not only secure but also efficient in terms of computational resources, power consumption, and memory usage. As the proliferation of connected devices continues to grow, the importance of lightweight cryptography becomes ever more critical.
\newline
This paper explores the current challenges and approaches in lightweight cryptography. We begin by discussing the fundamental principles and requirements for cryptographic algorithms in constrained environments. Following this, we delve into specific categories of lightweight cryptographic primitives, including authenticated encryption, hashing, and permutations. Each section provides an in-depth analysis of the state-of-the-art techniques, highlighting both their strengths and limitations.
\newline
By examining recent advancements and ongoing research, we aim to provide a comprehensive overview of the landscape of lightweight cryptography. This study serves as a valuable resource for researchers and practitioners seeking to understand the intricacies of cryptographic solutions tailored for resource-constrained devices.

