\section{Introduction}
Lightweight cryptography has emerged as a critical area of research, driven by the proliferation of resource-constrained devices such as IoT gadgets, RFID tags, and embedded systems. These devices require cryptographic solutions that are not only secure but also optimized for low power consumption, limited computational capability, and minimal memory usage. Traditional cryptographic algorithms often fall short in these environments due to their resource-intensive nature.
\newline
This paper addresses the challenges and approaches in the realm of lightweight cryptography. We explore the fundamental principles guiding the design of lightweight cryptographic algorithms and examine specific categories, including authenticated encryption, hashing, and permutations. Each section provides a comprehensive analysis of the current state-of-the-art techniques, discussing their advantages and limitations in the context of constrained environments.
\newline
Our goal is to present a detailed overview of the landscape of lightweight cryptography, highlighting recent advancements and identifying areas for future research. This study serves as a valuable resource for researchers and practitioners aiming to develop and implement efficient cryptographic solutions for the next generation of connected devices.
